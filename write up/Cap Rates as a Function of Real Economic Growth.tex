\begin{filecontents*}{example.eps}
gsave
newpath
  20 20 moveto
  20 220 lineto
  220 220 lineto
  220 20 lineto
closepath
2 setlinewidth
gsave
  .4 setgray fill
grestore
stroke
grestore
\end{filecontents*}

\RequirePackage{fix-cm}
\documentclass{svjour3}                     % onecolumn (standard format)
\smartqed  % flush right qed marks, e.g. at end of proof
\usepackage{graphicx}
\usepackage{mathptmx}      % use Times fonts if available on your TeX system


% insert here the call for the packages your document requires
\usepackage{natbib}
\usepackage{csvsimple}
\usepackage{breakcites}
\usepackage{caption}
\usepackage{blindtext}
\usepackage{amsmath}
\usepackage{float}
\usepackage{graphicx}
\usepackage{geometry}
\usepackage{longtable,array}
\usepackage{subfig}
\usepackage{graphicx}
\usepackage{url}
\usepackage{pdflscape}

\begin{document}
\title{x}
\date{}
\subtitle{x}
\titlerunning{x}        
\author{Matt Larriva, CFA         \and
        Peter Linneman, PhD.
}

%\authorrunning{Short form of author list} % if too long for running head

%\institute{M. Larriva\at
%              \email{mlarriva@fcpdc.com}           %  \\
%             \emph{Present address:} of F. Author  %  if needed
%           \and
%           P. Linneman\at
%              \email{plinneman@linnemanassociates.com} %  \\ 
%}

%\date{Received: date / Accepted: date}
% The correct dates will be entered by the editor

\maketitle

\begin{abstract}

\subsection*{Purpose}
To explain both national and market-level cap rate changes as simply an interaction between Gross Domestic Product and Consumer Price Index.

\subsection*{Design/Methodology/Approach}
We use a binary logistic regression with binary independent variables, trained on a synthetic minority over-sampling set, to explain cap rate expansion and contraction at the national level and at 20 MSAs.

\subsection*{Results}
Both the national model and the MSA models capture around 40\% of all cap rate expansion periods with a robust confusion matrix.

\subsection*{Originality}
This model contributes to the existing corpus by 1) establishing a statistically significant relationship between GDP, CPI, and cap rates which 2) holds explanatory power at the national and MSA levels, by 3) mapping the ground truth from scalar to binary.

\subsection*{Practical Implications}
The accuracy of the model suggests a simpler and more robust explanation for understanding cap rates and navigating property markets in an environment of fluctuating interest rates. 


\keywords{Binary Logistic Regression \and Cap Rates \and US Real Estate Markets \and Multifamily Cap Rate \and United States \and Apartment Cap Rate}

\end{abstract}

\pagebreak
\setlength{\parindent}{4em}
\setlength{\parskip}{1em}

\section{Introduction}
\label{intro}
At 16 trillion dollars, the value of commercial real estate in the United States \citep*{NAREITsize} represents half of a percent of the world's total wealth \citep*{CSsize}. And at 65\% home-ownership in the United States \citep*{fred_2020}, real estate represents a more invested-in asset class than equities--only 55\% of Americans own stock \citep*{saad_2020}. As such, the value of this asset class and the underlying determinants are of importance not only to owners and operators but also to the economy as a whole. 


\section{Literature Review}


\section{Data}
For the national model there are three data series that are used: the Consumer Price Index, the Gross Domestic Product, and the nominal Apartment cap rate series. 
The Gross Domestic Product is a quarterly, unadjusted (not de-seasoned) series from the St. Louis Fed's economic data page (FRED). It is important to use a nominal (versus real) series, as we are comparing this to the consumer price index, which is a close proxy to the interest rate series used to convert the GDP from nominal to real. That is to say, we would be double counting the impact of inflation.The Consumer Price Index series is sourced to the Bureau of Labor Statistics, and it is also a non-seasonally adjusted series. In both cases, we did not seek to alter the series with seasonal adjustments because the raw data itself contains information which we do not wish to strip. Knowing if the economy grew faster than inflation, even if it was in 4Q, and even if it was due to holiday sales, is valuable information. Such a situation suggests a very different environment from its seasonally adjusted counterpart which may shows an economy in which 4Q growth is similar to say the prior quarter.
Finally, cap rate series at the national level are sourced to Green Street. Green Street defines its cap rate series as the next-twelve-months' NOI divided by the spot asset value. 

MSA-level data follows the conventions of the national data. Non seasonally adjusted series are selected both for MSA GDP and the MSA's CPI. The former is from FRED while the later is from the Bureau of Labor Statistics.

Do We smooth the inputs?
 
\section{Selection, Specification, and Analysis Binary Logistic Regression}
 

\section{Results}

\subsection{Analysis of Results}


\section{Conclusion}

\pagebreak


% Authors must disclose all relationships or interests that 
% could have direct or potential influence or impart bias on 
% the work: 
%
\section*{Declarations}
\subsection{Funding}
This research is a part of one author's role as VP of Research and Data Analytics at a Real Estate Private Equity firm. 

\subsection{Conflict of interest}
One author works for a Real Estate Private Equity firm which has ownership interest in many office and multifamily assets throughout the US. 

\subsection{Availability of data and material}
Data available upon request.

\subsection{Code availability}
Code available upon request.

\subsection{Authors' contributions}
Each of the authors confirms that this manuscript has not been previously published and is not currently under consideration by any other journal.


% BibTeX users please use one of
%\bibliographystyle{spbasic2}      % basic style, author-year citations
\bibliographystyle{apalike}
%\bibliographystyle{spmpsci}      % mathematics and physical sciences
%\bibliographystyle{spphys}       % APS-like style for physics
%\bibliography{}   % name your BibTeX data base

% Non-BibTeX users please use
%\begin{thebibliography}{}
%
% and use \bibitem to create references. Consult the Instructions
% for authors for reference list style.
%
%\bibitem{RefJ}
% Format for Journal Reference
%Author, Article title, Journal, Volume, page numbers (year)
% Format for books
%\bibitem{RefB}
%Author, Book title, page numbers. Publisher, place (year)
% etc
%\end{thebibliography}
\bibliography{bib} 

\end{document}
% end of file template.tex

